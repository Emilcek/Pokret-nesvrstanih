\chapter{Dnevnik promjena dokumentacije}
		
		\textbf{\textit{Kontinuirano osvježavanje}}\\
				
		
		\begin{longtblr}[
				label=none
			]{
				width = \textwidth, 
				colspec={|X[2]|X[13]|X[3]|X[3]|}, 
				rowhead = 1
			}
			\hline
			\textbf{Rev.}	& \textbf{Opis promjene/dodatka} & \textbf{Autori} & \textbf{Datum}\\[3pt] \hline
			0.1 & Napravljen predložak.	& Sebastian Medjaković & 26.10.2023. 		\\[3pt] \hline 
			0.2	& Opis projekta i funkcionalni zahtjevi. & Sara Gašpar & 30.10.2023. 	\\[3pt] \hline
			0.3	& Nabrojeni obrasci uporabe & Sebastian Medjaković & 1.11.2023. 	\\[3pt] \hline
			0.4 & Opis obrazaca uporabe & svi & 1.11.2023. \\[3pt] \hline
			0.5 & Dodan model baze i nefunkcionalni zahtjevi & Sebastian Medjaković & 2.11.2023. \\[3pt] \hline 
			0.6 & Arhitektura i dizajn sustava & Sara Gašpar & 2.11.2023. \\[3pt] \hline 
			0.7 & revizija dokumentacije & Sara Gašpar, Sebastian Medjaković & 2.11.2023. \\[3pt] \hline 
			0.9 & Opisi obrazaca uporabe & * & 07.09.2013. \\[3pt] \hline 
			0.10 & Preveden uvod & * & 08.09.2013. \\[3pt] \hline 
			0.11 & Sekvencijski dijagrami & * & 09.09.2013. \\[3pt] \hline 
			0.12.1 & Započeo dijagrame razreda & * & 10.09.2013. \\[3pt] \hline 
			0.12.2 & Nastavak dijagrama razreda & * & 11.09.2013. \\[3pt] \hline 
			\textbf{1.0} & Verzija samo s bitnim dijelovima za 1. ciklus & * & 11.09.2013. \\[3pt] \hline 
			1.1 & Uređivanje teksta -- funkcionalni i nefunkcionalni zahtjevi & * \newline * & 14.09.2013. \\[3pt] \hline 
			1.2 & Manje izmjene:Timer - Brojilo vremena & * & 15.09.2013. \\[3pt] \hline 
			1.3 & Popravljeni dijagrami obrazaca uporabe & * & 15.09.2013. \\[3pt] \hline 
			1.5 & Generalna revizija strukture dokumenta & * & 19.09.2013. \\[3pt] \hline 
			1.5.1 & Manja revizija (dijagram razmještaja) & * & 20.09.2013. \\[3pt] \hline 
			\textbf{2.0} & Konačni tekst predloška dokumentacije  & * & 28.09.2013. \\[3pt] \hline 
			&  &  & \\[3pt] \hline	
		\end{longtblr}
	
	
		\textit{Moraju postojati glavne revizije dokumenata 1.0 i 2.0 na kraju prvog i drugog ciklusa. Između tih revizija mogu postojati manje revizije već prema tome kako se dokument bude nadopunjavao. Očekuje se da nakon svake značajnije promjene (dodatka, izmjene, uklanjanja dijelova teksta i popratnih grafičkih sadržaja) dokumenta se to zabilježi kao revizija. Npr., revizije unutar prvog ciklusa će imati oznake 0.1, 0.2, …, 0.9, 0.10, 0.11.. sve do konačne revizije prvog ciklusa 1.0. U drugom ciklusu se nastavlja s revizijama 1.1, 1.2, itd.}