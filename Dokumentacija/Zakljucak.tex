\chapter{Zaključak i budući rad}
		
		Zadatak naše grupe bio je razvoj web aplikacije pod nazivom "Wildtrack". Sama ideja aplikacije je olakšati korisnicima koordinaciju prilikom prolaženja i praćenja divljih životinja. Nakon 17 tjedana timskog rada, ostvaren je cilj te je projekt uspješno završen. Projekt je bio proveden kroz tri faze.
		
		Prva faza uključivala je okupljanje tima radi diskusije o općim konceptima za aplikaciju, izražavanje individualnih interesa i želja za sudjelovanjem u provedbi projekta te konačno raspodjelom zadataka za tu fazu. Kvalitetna provedba prve faze uvelike je olakšala daljni rad. Krenulo se sa intenzivnim radom na dokumentaciji. U sklopu toga izrađeni su obrasci i dijagrami (obrasci uporabe, sekvencijski i dijagrami razreda) koji su bili od pomoći članovima tima koji su zaduženi za razvoj frontenda odnosno backenda aplikacije. Prva faza  trajala je prvih 7 tjedana.
		
		U drugoj fazi fokus je bio na implementaciji same aplikacije gdje su svi članovi tima surađivali zajedno na programskom ostvarivanju projekta. Nedostatak iskustva većine članova tima prisilio je tim na neovisno učenje odabranih alata i programskih jezika kako bi ostvarili postavljene ciljeve. Osim što se radilo na implementaciji rješenja, u drugoj fazi projekta bilo je nužno napraviti dodatne UML dijagrame (stanja, aktivnosti, komponenti i razmještaja) te pripremiti pripadajuću dokumentaciju. Time smo značajno uštedjeli vrijeme tijekom razvoja aplikacije.
		
		Treća faza uključivala je testiranje sustava, pronalazak i ispravak grešaka, dorada izgleda aplikacije te implementaciji preostalih funkcionalnosti. Tijekom ove faze pojavilo se niz manjih zadataka koje su članovi tima preuzimali i obavljali samostalno.
		
		Članovi unutar time komunicirali su putem Whatsappa i Discorda, što je omogućilo brzu i efikasnu informiranost o napretku projekta među svim članovima grupe.
		Jednom tjedno svaki član tima bi prezentirao svoj napredak od prošlog puta i eventualno zatražio pomoć od ostatka tima ako je naišao na neki problem.
		
		Aplikaciju je moguće proširiti na način da tragači ostavljaju komentare voditelju kako im je bilo izvoditi akciju kod pojedinih istraživača, ali isto tako da istraživač šalje povratnu informaciju voditelju kako je zadovoljan s pojedinim tragačem.
		 
		Sudjelovanje u ovakvom projektu bilo je izuzetno vrijedno isukstvo za sve članove tima. Ovo je bio naš prvi ozbiljniji grupni projekt i uspješno smo se snašli u izazovima koje je projekt donio. Osim toga stekli smo uvid u važnost dobre vremenske organizacije i koordinacije među članovima tima. Konflikata unutar tima gotovo da nije ni bilo. Sudjelovanjem u više ovakvih projekata i boljim poznavanjem programskih alata projekt bi zasigurno bio kvalitetnije i brže ostvaren. Ovaj projekt predstavljao nam je prvi ozbiljniji susret s tehnologijama poput Gita i LateX-a. Tijekom razvoja web aplikacije, stekli smo iskustvo u korištenju modernih radnih okvira. Izrazito smo zadovoljni postignutim rezultatima i zajedničkim naporom tima koji je doveo do tih rezultata. 
		
	   
		
		
		
		
		
		
		
		
	
		
		 